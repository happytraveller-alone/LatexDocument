\documentclass{article}
\usepackage{fancyhdr} % Required for custom headers
\usepackage{lastpage} % Required to determine the last page for the footer
\usepackage{extramarks} % Required for headers and footers
\usepackage[usenames,dvipsnames]{color} % Required for custom colors
\usepackage{graphicx} % Required to insert images
\usepackage[fleqn]{amsmath}
\usepackage[cs4size,UTF8,winfonts,heading=true]{ctex}
\usepackage{tikz,pgfkeys,pgf,pgfplots}
\usepackage{times}
\usepackage{floatrow}
\usepackage[labelfont=bf,labelsep=quad]{caption}
\setlength{\parindent}{4em}
\addtolength{\parskip}{3pt}
\usepackage[compact]{titlesec}         % you need this package
\titlespacing*{\subsection}{0pt}{0pt}{0pt} % this reduces space between (sub)sections to 0pt, for example

% Margins
\topmargin=-0.45in
\evensidemargin=0in
\oddsidemargin=0in
\textwidth=6.5in
\textheight=9.0in
\headsep=0.25in
\headheight=13pt

\linespread{1.1} % Line spacing

% Set up the header and footer
\pagestyle{fancy}
\lhead{网络安全一班} % Top left header
\chead{\hmwkTitle} % Top center head
\rhead{3019244283 谢远峰 } % Top right header
\renewcommand\headrulewidth{0.4pt} % Size of the header rule
\renewcommand\footrulewidth{0.4pt} % Size of the footer rule
\renewcommand{\normalsize}{\fontsize{12pt}{\baselineskip}\selectfont}

%--Edit Title, Author and Date here
\newcommand{\hmwkTitle}{编译原理-语法分析} % Assignment title
\newcommand{\hmwkClassTime}{\today} % Class/lecture time
\newcommand{\hmwkAuthorName}{网安一班\ 3019244283\ 谢远峰 } % Your name


\title{
\vspace{2in}
\Huge{\textbf{\hmwkTitle}}\\
\vspace{2.5in}
}
\author{\textbf{\hmwkAuthorName}}
\date{\hmwkClassTime}

%------------------------------------------------------------------

\begin{document}

\maketitle
\newpage
\noindent
考虑文法$G_{2}[S]:\quad S \rightarrow a|\&|(  \qquad T \rightarrow TS|S $\\
(1)消除$G_2$左递归,得到(不含左递归)的$G_2^{'}$文法\\
(2)\ $G_2^{'}$是不是LL(1)的?如果是,给出证明,如果不是,将其改造成LL(1)的\\
(3)给出$G_2^{'}$的预测分析表\\
(4)给出符号串(a\&a)的预测分析表\\ \\
\noindent
(1)
\begin{gather}
    S \rightarrow a|\&|( \\
    T \rightarrow aT^{'}|\&T^{'}|(T)T^{'}\\
    T^{'} \rightarrow aT^{'}|\&T^{'}|(T)T^{'}|\varepsilon
\end{gather}
(2)\ $G_2^{'}$是LL(1)的,证明如下:

$\bullet $文法不含有左递归\\

$\bullet $对于文法中每一个非终结符 A 的各个产生式的候选首符集两两不相交

$FIRST(a)=\{a\}\neq FIRST(\&)=\{\&\}\neq FIRST((T))=\{(\}$

$FIRST(aT^{'})=\{a\}\neq FIRST(\&T^{'})=\{\&\}\neq FIRST((T)T^{'})=\{(\}$

$FIRST(aT^{'})=\{a\}\neq FIRST(\&T^{'})=\{\&\}\neq FIRST((T)T^{'})=\{(\}\neq FIRST(\varepsilon)$\\

$\bullet $对文法中的每个非终结符A ,若它存在某个候选首符集包含ε,则FIRST(A)∩FOLLOW(A)=Φ

$FIRST(T^{'})=\{a,\&,(,\varepsilon\}\bigcap FOLLOW(T^{'})=\{)\}=\phi $\\
(3)\\
FIRST SET:

$FIRST(a)=\{a\};FIRST(\&)=\{\&\};FIRST((T))=\{(\};FIRST(\varepsilon)=\{\varepsilon \};$

$FIRST(\&T^{'})=\{\&\};FIRST((T)T^{'})=\{(\};FIRST(aT^{'})=\{a\};$

$FIRST(S)=\{a,\&,(\};FIRST(T)=\{a,\&,(\};FIRST(T^{'})=\{a,\&,(,\varepsilon \}$\\
FOLLOW SET:

$FOLLOW(S)=\{\# \};FOLLOW(T)=\{\#,)\};FOLLOW(T^{'})=\{\#,)\}$
\begin{center}
    \begin{tabular}[h]{|c|c|c|c|c|c|}\hline
        ~       & a                          & \&                          & (                            & )                               & \#                              \\\hline
        S       & $S \rightarrow a$          & $S \rightarrow \&$          & $S \rightarrow (T)$          & ~                               & ~                               \\\hline
        T       & $T \rightarrow aT^{'}$     & $T \rightarrow \&T^{'}$     & $T \rightarrow (T)T^{'}$     & ~                               & ~                               \\\hline
        $T^{'}$ & $T^{'} \rightarrow aT^{'}$ & $T^{'} \rightarrow \&T^{'}$ & $T^{'} \rightarrow (T)T^{'}$ & $T^{'} \rightarrow \varepsilon$ & $T^{'} \rightarrow \varepsilon$ \\\hline
    \end{tabular}
\end{center}
\newpage
(4)
\begin{center}
    \begin{tabular}{|c|c|c|c|}\hline
        步骤 & 符号栈      & 输入串   & 所用产生式                      \\\hline
        0    & $\#S$       & $(a\&a)$ &                                 \\\hline
        1    & $\#)T($     & $(a\&a)$ & $S \rightarrow (T)$             \\\hline
        2    & $\#)T$      & $a\&a)$  &                                 \\\hline
        3    & $\#T^{'}a$  & $a\&a)$  & $T \rightarrow aT^{'}$          \\\hline
        4    & $\#T^{'}$   & $\&a)$   &                                 \\\hline
        5    & $\#T^{'}\&$ & $\&a)$   & $T^{'} \rightarrow \&T^{'}$     \\\hline
        6    & $\#T^{'}$   & $a)$     &                                 \\\hline
        7    & $\#T^{'}a$  & $a)$     & $T^{'} \rightarrow aT^{'}$      \\\hline
        8    & $\#T^{'}$   & $)$      &                                 \\\hline
        9    & $\#)$       & $)$      & $T^{'} \rightarrow \varepsilon$ \\\hline
        10   & $\#$        & $\#$     & ~                               \\\hline
    \end{tabular}
\end{center}
\end{document}