\documentclass[UTF8,fleqn]{article}
\usepackage[UTF8]{ctex}
\usepackage[a3paper, margin=1in,landscape]{geometry}
\usepackage[fleqn]{amsmath}
\usepackage{fancyhdr} % Required for custom headers
\usepackage{lastpage} % Required to determine the last page for the footer
\usepackage{extramarks} % Required for headers and footers
\usepackage{tikz}
\usetikzlibrary{shapes.geometric, arrows}
\usetikzlibrary{
  arrows.meta,
  calc,
  fit,
  positioning,
  quotes
}

\pagestyle{fancy}
\lhead{网络安全一班} % Top left header
\chead{\hmwkClass\ \hmwkTitle} % Top center head
\rhead{3019244283 谢远峰 } % Top right header
\lfoot{\lastxmark} % Bottom left footer
\cfoot{第\ \thepage\ 页} % Bottom center footer
% \rfoot{第\ \thepage\ 页} % Bottom right footer
\newcommand{\hmwkTitle}{信息安全数学基础} % Assignment title
\newcommand{\hmwkClass}{} % Course/class
\large
\title{Homework1}
\begin{document}
\begin{titlepage}
	\begin{center}
		\line(1,0){300}\\
		[0.65cm]
		\huge{\bfseries Homework II }\\
		\line(1,0){300}\\
		% \textsc{\Large Chapter 1}\\
		\textnormal{\Large \today}\\
		[5.5cm]
	\end{center}
	\begin{flushright}
		\texttt{\Large 谢远峰\\网安一班\\3019244283}\\
		[0.5cm]
	\end{flushright}
\end{titlepage}
\clearpage

% \begin{document}
% \newpage
% \end{document}
% \section*{习题1}
% 证明:如果素数p=4n+1,且d|n,则$\bigg{(}\frac{d}{p}\bigg{)}$  \\
% \begin{align*}
% 	\begin{split}
% 		(d,p)=1 \rightarrow &d^{p-1}\equiv 1(mod\ p) \qquad \text{费马小定理}\\
% 		\rightarrow &d^{4n}\equiv 1(mod\ p)\\
% 		\rightarrow &(d^{2n})^{2}\equiv 1^2(mod\ p)\qquad \text{同余性质}\\
% 		\rightarrow &d^{2n}\equiv 1(mod\ p)\\
% 		\rightarrow &d^{\frac{p-1}{2}}\equiv 1(mod\ p)\qquad \text{二次剩余理论}\\
% 	\end{split}
% \end{align*}

% \qquad d是模p的


% \vspace{3cm}
% \tikzstyle{startstop1} = [rectangle, rounded corners, minimum width=2.7cm, minimum height=1cm,text centered,text width=3cm, draw=black, fill=yellow!30]
% \tikzstyle{startstop2} = [rectangle, rounded corners, minimum width=2.7cm, minimum height=1cm,text centered,text width=3cm, draw=black, fill=orange!30]
% \tikzstyle{startstop3} = [rectangle, rounded corners, minimum width=2.7cm, minimum height=1cm,text centered,text width=3cm, draw=black, fill=red!30]
% \tikzstyle{startstop4} = [rectangle, rounded corners, minimum width=2.7cm, minimum height=1cm,text centered,text width=3cm, draw=black, fill=green!30]
% \tikzstyle{startstop5} = [rectangle, rounded corners, minimum width=2.7cm, minimum height=1cm,text centered,text width=3cm, draw=black, fill=brown!30]
% % \tikzstyle{io} = [trapezium, trapezium left angle=70, trapezium right angle=110, minimum width=3cm, minimum height=1cm, text centered, draw=black, fill=blue!30]
% % \tikzstyle{process} = [rectangle, minimum width=3cm, minimum height=1cm, text centered, draw=black, fill=orange!30]
% % \tikzstyle{decision} = [diamond, minimum width=3cm, minimum height=1cm, text centered, draw=black, fill=green!30]
% \tikzstyle{arrow} = [thick,->,>=stealth]
% \begin{tikzpicture}[node distance=1.8cm]
% 	\node (Susceptible) [startstop1] {Susceptible易感人群};
% 	\node (Exposed) [startstop1, left of= Susceptible, xshift=-1.8cm] {$\beta $};
% 	\node (Exposed) [startstop2, below of= Susceptible, yshift=-0.2cm] {Exposed\\传染人群};
% 	\node (Infectious) [startstop3, below of=Exposed, yshift=-0.2cm] {Infectious感染人群};
% 	\node (Rocovered) [startstop4, right of= Infectious , xshift=1.8cm] {Rocovered\\康复人群};
% 	\node (Dead) [startstop5, below of= Infectious , yshift=-0.1cm] {Dead\\死亡人群};

% 	\draw [arrow] (Susceptible) -- node[anchor=east] {$\beta$} (Exposed);
% 	\draw [arrow] (Exposed) -- node[anchor=east] {$\sigma$} (Infectious);
% 	\draw [arrow] (Infectious) -- node[anchor=south] {$\gamma$} (Rocovered);
% 	\draw [arrow] (Infectious) -- node[anchor=east] {$\mu$} (Dead);

% \end{tikzpicture}

\makeatletter
\tikzset{
	meta box/.style={
			draw,
			black,
			very thick,
			text centered
		},
	punkt/.style={
			meta box,
			rectangle,
			rounded corners,
			inner sep=2.5pt,
			minimum height=1em,
			minimum width=3em,
			align=center,
			text width=5em
		},
	round box/.style={
			meta box,
			circle
		},
	every fit/.style={
			draw,
			thick,
			dashed,
			gray,
			inner sep=10pt
		}
}

% helper macro
\newcommand\tikz@expand@dimen[2]{\tikzset{minimum #2=#1}}
% tikz styles: adjust min width/height of nodes
\tikzset{
	add dimen/.code 2 args={%
			\pgfkeysgetvalue{/pgf/minimum #1}\tikz@dimen@min
			\expandafter\tikz@expand@dimen\expandafter{\tikz@dimen@min + #2 * 2em}{#1}%
		},
	wider/.style={add dimen={width}{#1}},
	higher/.style={add dimen={height}{#1}},
}
\makeatother
\begin{tikzpicture}[font=\LARGE, node font=\LARGE]
	%% rectangles
	\node[punkt, wider=1, higher=1]
	(start) {S-E-I-R-D\\ $\beta,\sigma,\gamma,\mu$};
	\node[punkt, below=2 of start]
	(S) {Susceptible\\易感人群};
	\node[punkt, below=1 of S,label=right:传染人数]
	(beta) {$\beta = R_0\alpha$};
	\node[round box, xshift=-4cm,below=0.9 of S,label=above:聚集率]
	(delta) {$\delta$};
	\node[punkt, wider=1, below=1 of beta]
	(E) {Exposed\\传染人群};
	\node[round box,below=1.8 of delta,label=below:接种率]
	(theta) {$\theta$};
	\node[round box, below=1 of E,label=right:发病率]
	(sigma) {$\sigma$};
	\node[punkt, below=1 of sigma]
	(I) {Infectious\\感染人群};
	\node[round box, xshift=3cm , below=1.5 of I,label=right:康复率]
	(gamma) {$\gamma$};
	\node[round box, xshift=-3cm , below=1.5 of I,label=left:死亡率]
	(mu) {$\mu$};
	\node[punkt, wider=1, higher=1, below=1 of gamma]
	(R) {Recovered\\康复人群};
	\node[punkt, wider=1, higher=1, below=3.75 of mu]
	(D) {Dead\\死亡人群};


	\begin{scope}[xshift=12cm]
		\node[punkt, wider=1, higher=1]
		(equation5) {$N = S + E + I + R + D$};
		\node[punkt, wider=1, higher=1, below=1.5 of equation5]
		(equation1) {$\frac{dS}{dt}= -\frac{\beta SI}{N}$};
		\node[punkt, wider=1, higher=1, below=2.5 of equation1]
		(equation2) {$\frac{dE}{dt}= \frac{\beta SI}{N} - \sigma E$};
		\node[punkt, wider=1, higher=1, below=2.8 of equation2]
		(equation3) {$\frac{dI}{dt}= \sigma E - \gamma I - \mu I$};
		\node[punkt, wider=1, higher=1, below=3.6 of equation3]
		(equation4) {$\frac{dR}{dt}= \gamma I$};
		\node[punkt, wider=1, higher=1, below=0.8 of equation4]
		(equation6) {$\frac{dD}{dt}= \mu I$};
	\end{scope}

	\begin{scope}[xshift=18cm,yshift=-1cm]
		\node[punkt, wider=1, higher=1]
		(init1) {$N\ = 2830000$};
		\node[punkt, wider=1, higher=1, below=0.5 of init1]
		(init2) {$S\ = 2839970$};
		\node[punkt, wider=1, higher=0.8, below=0.5 of init2]
		(init3) {$E\ =\ 10$};
		\node[punkt, wider=1, higher=0.8, below=0.5 of init3]
		(init4) {$I\ =\ 20$};
		\node[punkt, wider=1, higher=0.8, below=0.5 of init4]
		(init5) {$R\ =\ 0$};
		\node[punkt, wider=1, higher=0.8, below=0.5 of init5]
		(init6) {$D\ =\ 0$};
	\end{scope}

	\begin{scope}[xshift=22.5cm,yshift=-2cm]
		\node[punkt, wider=1, higher=0.8]
		(init7) {$\beta \ =\ 4$};
		\node[punkt, wider=1, higher=0.8, below=0.5 of init7]
		(init8) {$\sigma \ =\ 0.0012$};
		\node[punkt, wider=1, higher=1, below=0.5 of init8]
		(init9) {$\gamma \ =\ 0.006$};
		\node[punkt, wider=1, higher=1, below=0.5 of init9]
		(init10) {$\mu \ =\ 0.0001$};
		\node[punkt, wider=1, higher=1, below=0.5 of init10]
		(init11) {$days\ =\ 367$};
	\end{scope}

	\begin{scope}[xshift=18cm,yshift=-16cm]
		\node[punkt, wider=1, higher=0.8]
		(change1) {$\beta \ =\ 6$};
		\node[punkt, wider=1, higher=0.8, below=0.5 of change1]
		(change2) {$\sigma \ =\ 0.002$};
	\end{scope}

	\begin{scope}[xshift=22.5cm,yshift=-16cm]
		\node[punkt, wider=1, higher=0.8]
		(change4) {$\sigma \ =\ 0.0008$};
		\node[punkt, wider=1, higher=0.8, below=0.5 of change4]
		(change5) {$\gamma \ =\ 0.01$};
		\node[punkt, wider=1, higher=0.8, below=0.5 of change5]
		(change6) {$\mu \ =\ 0.00005$};
	\end{scope}

	% %% arrowed lines
	\path[arrows={->[scale=1.1]}, thick]
	% (E)         edge (pool)
	(start)      edge ["input"](S)
	(S)          edge (beta)
	(delta)      edge (beta)
	(theta)      edge (beta)
	(theta)      edge (sigma)
	(beta)       edge (E)
	(E)          edge (sigma)
	(sigma)      edge (I)
	(I)          edge (mu)
	(I)          edge (gamma)
	(gamma)      edge (R)
	(mu)         edge (D)
	(start)      edge ["模型人口总和"](equation5)
	(S)          edge ["   单日受影响人群"](equation1)
	(E)          edge ["   单日受传染人群"](equation2)
	(I)          edge ["单日发病人群 "](equation3)
	(R)          edge ["单日恢复人群"](equation4)
	(D)          edge ["单日死亡人群"](equation6);

	\node[
		fit=(S) (beta) (E) (sigma) (I) (mu) (gamma)(D)(R),
		label={[gray, anchor=south]north west:传染病模型}] {};
	\node[
		fit=(equation1)(equation2)(equation3)(equation4)(equation6),
		label={[gray, anchor=south]north :模型微分方程组}] {};
	\node[
		fit=(init1)(init2)(init3)(init4)(init5)(init6)(init7)
		(init8)(init9)(init10)(init11),
		label={[gray, anchor=south]north :初始化变量}] {};
	\node[
		fit=(change1)(change2),
		label={[gray, anchor=south]north :人口聚集改变}] {};
	\node[
		fit=(change4)(change5)(change6),
		label={[gray, anchor=south]north :接种改变}] {};
\end{tikzpicture}

\newpage
\begin{tikzpicture}[font=\LARGE, node font=\LARGE]
	%% rectangles
	\node[punkt, wider=1, higher=1]
	(E) {input: $E\colon (f \times k)$};
	\node[meta box, below=2 of E]
	(pool) {max pool};
	\node[punkt, wider=1, below=1.5 of pool]
	(Z) {$Z\colon (f \times 1)$};
	\node[punkt, below=2 of Z]
	(Z1) {$(f/r \times 1)$};
	\node[punkt, wider=1, below=2 of Z1]
	(A) {$A \colon (f \times 1)$};
	\node[round box, label=right:Re-Scale, below=of A]
	(multiply) {$\times$};
	\node[punkt, wider=1, higher=1, below=of multiply]
	(V) {$V \colon (f \times k)$};

	\begin{scope}[xshift=8cm, yshift=-4cm]
		\node[punkt, wider=1, higher=2.5]
		(bilinear1) {$\forall i<j$ \\ $e_i \cdot W \odot e_j \to p_{i,j} \in R^k$};
		\node[punkt, wider=1, higher=2.5, below=5 of bilinear1]
		(bilinear2) {$\forall i<j$ \\ $v_i \cdot W \odot v_j \to q_{i,j} \in R^k$};
	\end{scope}

	\node[punkt, wider=1.7, right=4 of bilinear1]
	(P) {$p$: $(n \times k)$};
	\node[punkt, wider=1.7, right=4 of bilinear2]
	(Q) {$q$: $(n \times k)$};
	\node[round box, label=left:concat]
	(add) at ($(P)!0.5!(Q)$){$+$};
	\node[punkt, rounded corners=0pt, wider=2, higher=2.5, right=5 of add]
	(dnn) {DNN};

	%% arrowed lines
	\path[arrows={->[scale=1.1]}, thick]
	(E)         edge (pool)
	(pool)      edge (Z)
	(Z)         edge["${} \cdot W_1 \in R^{f\times f/r}$"'{right=3pt, inner sep=0pt, name=tip1}] (Z1)
	(Z1)        edge["${} \cdot W_2 \in R^{f/r \times f}$"'{right=3pt, inner sep=0pt, name=tip2}] (A)
	(A)         edge (multiply)
	(multiply)  edge (V)
	(E)         edge[bend right=90] (multiply)
	(E)         edge (bilinear1)
	(bilinear1) edge (P)
	(P)         edge (add)
	(V)         edge (bilinear2)
	(bilinear2) edge (Q)
	(Q)         edge (add)
	(add)       edge["$c\colon (2n \times k)$"] (dnn)
	(dnn.east)  edge["output"] +(3, 0);

	%% frames
	\node[
		fit=(pool) (Z) (multiply) (tip1) ($(Z1)!-1!(tip2.east)$),
		label={[gray, anchor=south]north west:SENET}] {};
	\node[
		fit=(bilinear1) (bilinear2),
		label={[gray, anchor=south]north west:Bilinear Interaction Layer}] {};
\end{tikzpicture}

\newpage
\vspace{2cm}
\begin{tikzpicture}[font=\LARGE, node font=\LARGE]
	%% rectangles
	\node[punkt, wider=1, higher=1]
	(start1) {Disease Decline\\$R_0 = 0$};
	\node[meta box, below=2 of start1]
	(patient1) {Patient 0};
	\node[round box, below=1.5 of patient1]
	(none) {X};

	\begin{scope}[xshift=8cm]
		\node[punkt, wider=1, higher=1]
		(start2) {Disease Decline\\$R_0 = 1$};
		\node[meta box, below=2 of start2]
		(patient2) {Patient 0};
		\node[meta box, below=1.5 of patient2]
		(patient3) {Patient 1};
		\node[meta box, below=1.5 of patient3]
		(patient4) {Patient 2};
	\end{scope}

	\begin{scope}[xshift=20cm]
		\node[punkt, wider=1, higher=1]
		(start3) {Disease Decline\\$R_0 = 2$};
		\node[meta box, below=2 of start3]
		(patient1_1) {Patient 0};
		\node[meta box,xshift = -3cm, below=1.5 of patient1_1]
		(patient2_1) {Patient 1};
		\node[meta box,xshift = 3cm, below=1.5 of patient1_1]
		(patient2_2) {Patient 2};
		\node[meta box,xshift = -1.5cm, below=1.5 of patient2_1]
		(patient3_1) {Patient 3};
		\node[meta box,xshift = 1.5cm, below=1.5 of patient2_1]
		(patient3_2) {Patient 4};
		\node[meta box,xshift = -1.5cm, below=1.5 of patient2_2]
		(patient3_3) {Patient 5};
		\node[meta box,xshift = 1.5cm, below=1.5 of patient2_2]
		(patient3_4) {Patient 6};
	\end{scope}

	% arrowed lines
	\path[arrows={->[scale=1.1]}, thick]
	(patient1)  edge(none)
	(patient2)  edge(patient3)
	(patient3)  edge(patient4)
	(patient1_1)edge(patient2_1)
	(patient1_1)edge(patient2_2)
	(patient2_1)edge(patient3_1)
	(patient2_1)edge(patient3_2)
	(patient2_2)edge(patient3_3)
	(patient2_2)edge(patient3_4);
	% frames
	\node[
		fit=(start1)(patient1)(none),
		label={[gray, anchor=south]north west:1}] {};
	\node[
		fit=(start2)(patient2)(patient3)(patient4),
		label={[gray, anchor=south]north west:2}] {};
	\node[
		fit=(start3)(patient1_1)(patient2_1)(patient2_2)
		(patient3_1)(patient3_2)(patient3_3)(patient3_4),
		label={[gray, anchor=south]north west:3}] {};
\end{tikzpicture}

\clearpage
\begin{tikzpicture}[font=\LARGE, node font=\LARGE]
	%% rectangles
	\node[punkt, wider=1, higher=1]
	(start) {程序后端};
	\node[punkt, wider=1, higher= 1,below=3 of start]
	(S) {编码器};
	\node[punkt, wider=1, higher= 1,below=15 of S]
	(S) {编码器};

	\begin{scope}[xshift=18cm]
		\node[punkt, wider=1, higher=1]
		(equation5) {程序前端};
		\node[punkt, wider=1, higher=1, below=3 of equation5]
		(equation1) {电机};
	\end{scope}


	% %% arrowed lines
	\path[arrows={->[scale=1.1]}, thick]
	% (E)         edge (pool)
	(start)      edge ["发送指令"](S)
	(S)          edge ["接收数据"](start)
	(start)      edge ["程序登录、电机连接、工作状态、纪念币接收"](equation5)
	(equation5)          edge ["印花图案选择、消息接收的确认"](start)
	(S)          edge ["编码器驱动电机"](equation1)
	(equation1)  edge ["电机提供编码器数据"](S);

\end{tikzpicture}
\end{document}