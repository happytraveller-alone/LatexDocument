\documentclass[UTF8]{article}
\usepackage[UTF8]{ctex}
\usepackage[a4paper, margin=1in]{geometry}
% \usepackage{minted}
\usepackage{amsmath}
% \usepackage{cite}
\large
\title{Homework-Chapter 1}
\begin{document}
\begin{titlepage}
    \begin{center}
        \line(1,0){300}\\
        [0.65cm]
        \huge{\bfseries Homework I }\\
        \line(1,0){300}\\
        \textsc{\Large Chapter 1}\\
        \textnormal{\Large \today}\\
        [5.5cm]
    \end{center}
    \begin{flushright}
        \texttt{\Large 谢远峰\\网安一班\\3019244283}\\
        [0.5cm]
    \end{flushright}
\end{titlepage}

\section*{Question 1}
What advantage does a circuit-switched network have over a packet-switched network? \par
电路交换的优点:通信时延小,有序传输,无排队冲突,适用范围广,实时性强,控制简单。可在段响应时间内保证定量的端到端带宽传输。\par
What advantages does TDM have over FDM in a circuit-switched network? \par
FDM(Frequency-Division Multiplexing,频分复用) TDM(Time-Division Multiplexing,时分复用),TDM能高效地使用网络,冲突性较低,适用于数字信号和模拟信号,构造简单,容错率较低。FDM只适用于模拟信号,需要硬件设置将信号转换为合适的频率。
\section*{Question 2}
\subsection*{a}
When circuit switching is used, how many users can be supported?\par
两个用户,每一个用户占用1Mbps的链接带宽
\subsection*{b}
Why will there be essentially no queuing delay before the link if two or fewer users transmit at the same time?Why will there be a queuing delay if three users transmit at the same time? \par
每个用户传输需求为1Mbps,总带宽为2Mbps,因此两人及以下不会产生队列延迟;若有N个用户,其不发生队列时延的理想总带宽为(2N)Mbps>2Mbps,因此会发生时延。
\subsection*{c}
Find the probability that a given user is transmitting. ——0.2(20\%)
\subsection*{d}
Find the probability that at any given time, all three users are transmitting simultaneously. Find the fraction of time during which the queue grows.\par
$(0.2)^3= 0.008 = 0.8\%$ ,队列增长的时间比率也为0.008( $0.8\%$ )
\section*{Question 3}
\subsection*{a}
Express the propagation delay, dprop in terms of m and s. ——$d_{drop}=m/s(s)$
\subsection*{b}
Determine the transmission time of the packet, dtrans, in terms of L and R. ——$d_{tras}=L/R(s)$
\subsection*{c}
Ignoring processing and queuing delays, obtain an expression for the end-to-end delay. \par $d_{eToe}=d_{drop}+d_{tras}=m/s+L/R(s)$
\subsection*{d}
Suppose Host A begins to transmit the packet at time t = 0. At time t = dtrans, where is the last bit of the packet?——刚刚离开A
\subsection*{e}
Suppose dprop is greater than dtrans. At time t = dtrans, where is the first bit of the packet?——在链路上,并且尚未达到主机B
\subsection*{f}
Suppose dprop is less than dtrans . At time t = dtrans, where is the first bit of the packet?——已经到达主机B
\subsection*{g}
Suppose $s = 2.5\times 10^8$ m/s, L = 1500 bytes, and R = 10 Mbps. Find the distance m so that dprop equals dtrans.\par
\begin{gather}
    d_{prop} = d_{tras} \\
    m/s = L/R \\
    m = Ls/R = 2.5*10^{8}*1500*8/10*10^6bps = 3*10^5m
\end{gather}
\section*{Question 4}
\subsection*{a}
What is the packet inter-arrival time at the destination? That is, how much time elapses from when the last bit of the first packet arrives until the last bit of the second packet arrives?——$L/R_s$
\subsection*{b}
Is it possible that the second packet queues at the input queue of the second link? Explain. Now suppose that the server sends the second packet T seconds after sending the first packet. How large must T be to ensure no queuing before the second link? Explain.\par
可能,第二个分组在第一个分组未完全输入到第二条链路前到达,P1的时间为$T_1=L/R_s+L/R_s+d_{prop}$第二个分组到达路由器的时间为$T_2=2L/R_s+d_{prop}$ \par
使拥挤的情况不发生,即$T_2+T > T_1 \rightarrow L/R_s+T>L/R_c \rightarrow T > L/R_c-L/R_s$
$R1 := \Pi_{maker}, model(Product \bowtie PC) $
\end{document}